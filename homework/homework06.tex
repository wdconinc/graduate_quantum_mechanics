\documentclass[letterpaper,11pt]{article}
\usepackage[utf8x]{inputenc}
\usepackage{enumerate}
\usepackage{enumitem}
\usepackage{fullpage}
\usepackage{amsmath}
\usepackage{hyperref}

\usepackage{pgf}
\usepackage{tikz}
\usetikzlibrary{arrows,shapes,trees}

%opening
\title{Physics 621 (Fall 2017) \\ Homework Assignment 6}
\date{Due: Wednesday October 18, 2017}

\begin{document}

\maketitle

\paragraph*{Symmetries}

All of the following questions are Category C assignment questions; you are welcome to work with others but you should submit your own work.

\begin{enumerate}
  \item Wave packet spreading. (LB 9.7.2, minor modifications)
  \begin{enumerate}
    \item Let $\langle X^2 \rangle(t)$ be the mean square position of the state $|\phi(t)\rangle$. Show that
    $$ \frac{d}{dt} \langle X^2 \rangle(t) = \frac{1}{m} \langle PX + XP \rangle = \frac{i\hbar}{m} \int_{-\infty}^{+\infty} dx x \left[ \phi \frac{\partial \phi^*}{\partial x} - \phi^* \frac{\partial \phi}{\partial x} \right]. $$
    \item Show that if the particle is free, $V(x) = 0$, then
    $$ \frac{d^2}{dt^2} \langle X^2 \rangle(t) = \frac{2}{m^2} \langle P^2 \rangle = 2 v_1^2 = \mbox{const}, $$
    with $v_1^2 = \langle P^2 \rangle / m^2$.
    \item Derive
    $$ \langle X^2 \rangle(t) = \langle X^2 \rangle(t = 0) + \xi_0 t + v_1^2 t^2, \quad \xi_0 = \left. \frac{d}{dt} \langle X^2 \rangle\right|_{t = 0}, $$
    as well as
    $$ \left(\Delta x(t)\right)^2 = \left(\Delta x(t = 0)\right)^2 + \left[ \xi_0 - 2 v_0 \langle X \rangle (t = 0) \right] t + (v_1^2 - v_0^2) t^2, $$
    with $v_0 = \langle P \rangle / m = \mbox{const}$.
  \end{enumerate}
  \item Gaussian wave packets. (LB 9.7.3, minor modifications)
  \begin{enumerate}
    \item We assume that the wave function $\tilde\phi(k)$ is a Gaussian in momentum representation
    $$ \tilde\phi(k) = \frac{1}{(\pi \sigma^2)^{1/4}} \exp \left[ -\frac{(k-\bar{k})^2}{2 \sigma^2} \right]. $$
    Show that this wave function is properly normalized. What is the dispersion $\Delta k(t = 0)$?
    \item Calculate the position representation $\phi(x,t=0)$,
    $$ \phi(x,t = 0) = \frac{\sigma}{(\pi \sigma^2)^{1/4}} \exp \left[ i \bar{k} x - \frac{1}{2} \sigma^2 x^2 \right]. $$
    What is the dispersion $\Delta x(t = 0)$? Show that this satisfies the Heisenberg inequality for $x$ and $k$.
    \item Calculate $\phi(x,t)$ to first order in the stationary phase approximation and show that
    $$ \phi(x,t) = \exp\left( \frac{i\hbar\bar{k}^2}{2m} t \right) \phi(x - v_g t, 0), \quad v_g = \frac{\hbar \bar{k}}{m}.$$
    For what $t$ is this approximation valid?
    \item Calculate $\phi(x,t)$ exactly:
    $$ \phi(x,t) = \frac{\sigma'}{(\pi \sigma^2)^{1/4}} \exp \left[ i \bar{k} x - i \omega(\bar{k}) t - \frac{1}{2} \sigma'^2 (x - v_g t)^2 \right], $$
    with
    $$ \frac{1}{\sigma'^2} = \frac{1}{\sigma^2} + \frac{i\hbar t}{m}. $$
    What is the dispersion $\Delta x(t)$?
  \end{enumerate}
  \item The Lennard-Jones potential for atomic interactions. (LB 9.7.5)
  \begin{enumerate}
    \item The potential energy of two atoms separated by a distance $r$ is often well represented by the Lennard-Jones potential:
    $$ V(r) = \epsilon \left[ \left( \frac{\sigma}{r} \right)^{12} - 2 \left( \frac{\sigma}{r} \right)^{6} \right], $$
    where $\epsilon$ and $\sigma$ are parameters with the dimension of energy and length. Calculate the position $r_0$ of the potential minimum and sketch $V(r)$ qualitatively. Show that near $r = r_0$
    $$ V(r) \approx - \epsilon \left[ 1 - 36 \left( \frac{r - r_0}{r_0} \right)^2 \right] = \frac{1}{2} m \omega^2 (r - r_0)^2 + V_0. $$
    \item In the case of helium, $\epsilon \approx 10^{-1}$\,eV and $r_0 \approx 0.3$\,nm. Calculate the vibration frequency $\omega$ and the energy $\hbar \omega/2$ of the ground state. Why does helium remain a liquid even if the temperature $T \to 0$? Does this reasoning hold for the two isotopes $^3$He and $^4$He?
    \item For hydrogen, $\epsilon \approx 4$\,eV. Why does hydrogen become a solid at low temperatures? What about the noble gases (argon, neon, etc.)?
  \end{enumerate}
\end{enumerate}

\end{document}
